\documentclass[titlepage, letterpaper]{article}
\usepackage[utf8]{inputenc}
\usepackage{fancyhdr}
\usepackage{amsmath}
\usepackage{extramarks}
\usepackage{enumitem}
\usepackage{amssymb}
\usepackage{booktabs}
\usepackage{tcolorbox}
\usepackage{gensymb}
\usepackage{booktabs}
\usepackage{graphicx}
\usepackage{caption}

\topmargin=-0.45in
\evensidemargin=0in
\oddsidemargin=0in
\textwidth=6.5in
\textheight=9.0in
\headsep=0.25in
\setlength{\parskip}{1ex}
\setlength{\parindent}{0ex}

%
% You should change this things~
%

\newcommand{\mahclass}{Sistemas de Búsqueda y Razonamiento}
\newcommand{\mahtitle}{Guía de Estudio}
\newcommand{\mahdate}{\today}
\newcommand{\spacepls}{\vspace{5mm}}

%
% Header markings
%

\pagestyle{fancy}
\lhead{Guía de Estudio}
\chead{}
\rhead{}
\lfoot{}
\rfoot{}


\renewcommand\headrulewidth{0.4pt}
\renewcommand\footrulewidth{0.4pt}
\renewcommand{\familydefault}{\sfdefault} %The sans-serif font and the like

% Alias for the Solution section header
\newcommand{\solution}{\textbf{\large Solución}}

%Alias for the new step section
\newcommand{\steppy}[1]{\textbf{\large #1}}

%
% My actual info
%

\title{
\vspace{1in}
\textbf{Tecnológico de Monterrey} \\
\vspace{0.5in}
\textmd{\mahclass} \\
\vspace{0.5in}
\textsc{\mahtitle}
\author{01170065  - MIT \\
Xavier Fernando Cuauhtémoc Sánchez Díaz \\
\texttt{mail@gmail.com}}
\date{\mahdate}
}

\begin{document}

\begin{titlepage}
    \maketitle
\end{titlepage}

%
% Actual document starts here~
%

\section{Metaheurísticas}

Una heurística es una estrategia que \textit{guía} el proceso de búsqueda del óptimo.
Su objetivo es explorar de manera eficiente el espacio de búsqueda encontrando soluciones óptimas o cercanas a las óptimas.

Las heurísticas son usualmente específicas para un tipo de problema en cuestión, y explotan las particularidades de cada problema.
Son algoritmos aproximados y generalmente no son deterministas.

Las metaheurísticas no son específicas a los problemas,
e incorporan algún tipo de memoria para guiar la búsqueda hacia óptimos globales.

Las metaheurísticas son de naturaleza estocástica, adaptables y fácilmente paralelizables.

\subsection{Ant Colony Optimization} % (fold)
\label{sub:ants}

Se basan en el comportamiento natural de las hormigas:
\begin{tcolorbox}
\begin{enumerate}
    \item Las hormigas vagan libremente, buscando una solución.
    \item Al encontrar una solución, regresan al hormiguero y dejan marcado el camino.
    \item El camino atrae a más hormigas, quienes de manera estocástica deciden si siguen un camino más corto o el camino de antes.
\end{enumerate}
Al final, todos los caminos convergen en una solución aproximada muy cercana a la óptima.
\end{tcolorbox}

Los tipos de problema a los que se aplica son de agente viajero, asignación cuadrática, programación de tareas o satisfacción de restricciones.

Cada hormiga tiene un componente de probabilidad de transición, un rastro de feromona, un nivel de visión de los caminos, memoria (o conjunto tabú) y coeficientes de la importancia de escoger un camino corto o un camino atractivo.

\begin{equation}
\label{eq:ant_transition}
p^k_{ij}(t) =
\begin{cases}
\dfrac{[\tau_{ij}(t)]^{\alpha}[\eta_{ij}]^{\beta}}{\sum_{k \in \text{allowed}}[\tau_{ik}(t)]^{\alpha}[\eta_{ik}]^{\beta}}, & j \in \text{allowed}_k \\[2ex]
0, & j \not \in \text{allowed}_k
\end{cases}
\end{equation}

\begin{equation}
\label{eq:ant_pheromone}
    \tau_{ij}(t)
\end{equation}

\begin{equation}
\label{eq:ant_visibility}
\eta_{ij} = \dfrac{1}{d_{ij}}
\end{equation}

La Ecuación \ref{eq:ant_transition} es para calcular la \textbf{probabilidad de transición} para cada ciudad.
Ecuación~\ref{eq:ant_pheromone} representa el cálculo de feromonas para el tiempo $t$.
La Ecuación~\ref{eq:ant_visibility} representa la visibilidad de una ciudad dependiendo de la distancia a la que se encuentre.
$\alpha, \beta$ son los parámetros de cuán importante es seguir una ciudad atractiva (con feromonas) o una ciudad visible, respectivamente.

Específicamente, el rastro de feromonas se actualiza de este modo:

\begin{equation}
\label{eq:ant_phero_update_global}
    \tau_{ij}(t+n) = \rho \tau_{ij} + \Delta \tau_{ij}
\end{equation}

Donde $\rho$ es el parámetro de evaporación de la feromona, y $\Delta\tau_{ij} = \Delta\tau^1_{ij} + \Delta\tau^2_{ij} + \dots + \Delta\tau^m_{ij}$ representa los cambios de feromonas entre dos ciudades $i,j$ para las $m$ hormigas.

Este rastro entre dos ciudades se calcula usando la siguiente ecuación:

\begin{equation}
\label{eq:ant_phero_update_p_ant}
\Delta\tau^k_{i,j} =
\begin{cases}
\dfrac{Q}{L_k}, & (i,j) \in \text{tour}_k \\[2ex]
0, & (i,j) \not \in \text{tour}_k
\end{cases}
\end{equation}

donde $Q$ es el parámetro de cuánta feromona se agrega por hormiga al pasar, y $L_k$ es la longitud del tour de la $k$-ésima hormiga.

El criterio de paro para un algoritmo de hormigas puede ser estancamiento (muchas iteraciones sin mejora), o bien llegar a un número límite de iteraciones.

\subsection{Artificial Bee Colony} % (fold)
\label{sub:bees}

Así como las hormigas, la colonia artificial de abejas se basa en el comportamiento habitual de estos insectos.
A diferencia de las hormigas, que construyen la solución poco a poco, cada abeja representa una solución.

El algoritmo es básicamente el siguiente:

\begin{tcolorbox}
\begin{enumerate}
    \item Se generan $n$ abejas exploradoras con soluciones aleatorias.
    \item Se evalúa la aptitud de cada una
    \item Se seleccionan $m$ mejores abejas y $e$ abejas sobresalientes para buscar en los alrededores, ($e < m$).
    \item Se envían $p$ abejas para cada uno de los $m$ sitios, y $q$ abejas para cada uno de los $e$ sitios.
    \item Se selecciona la mejor abeja de cada uno de los $m+e$ sitios.
    \item Se agregan $n-m$ abejas nuevas de manera aleatoria.
    \item Repetir desde 3, mientras no se alcance criterio de paro.
\end{enumerate}

Este proceso se repite hasta que se alcance el criterio de paro: un número máximo de iteraciones o bien estancamiento de la solución.
\end{tcolorbox}

Es importante determinar un tamaño máximo de vecindario, para enviar a las abejas que van a los $m,e$ sitios a buscar CERCA de donde estaban las demás abejas.
Este parámetro suelen llamarlo \textbf{ngh}, apócope de \textit{neighborhood}.

Este algoritmo encuentra soluciones óptimas y supera el problema de quedar estancado en óptimos locales.
Sin embargo al igual que con las hormigas, hay muchos parámetros que deben ajustarse manualmente.

\subsection{Particle Swarm Optimization} % (fold)
\label{sub:pso}

El algoritmo de optimización de enjambre de partículas es también bio-inspirado, pero esta vez en aves migratorias.

Pensado para un espacio $n$-dimensional, cada partícula (o cada pato) tiene una \textbf{velocidad} y una \textbf{posición}, y es tratada como un vector.
Cada individuo recuerda la mejor posición en la que ha estado, y sabe la mejor posición que han visitado sus vecinos más próximos.

Tras crear y posicionar aleatoriamente a la parvada, el algoritmo es básicamente el siguiente:

\begin{tcolorbox}
\begin{enumerate}
    \item Calcular la aptitud de cada partícula.
    \item Actualizar el mejor personal (\textbf{pbest}) si su posición actual es mejor que lo que recuerda.
    \item Escoger a la partícula con la mejor aptitud.
    \item Calcular la nueva velocidad de todas las partículas.
    \item Calcular la nueva posición de todas las partículas.
    \item Repetir todo mientras no se alcance un criterio de paro.
\end{enumerate}
\end{tcolorbox}

Es importante mencionar que el tamaño del vecindario debe ser previamente definido.

La ecuación para actualizar la velocidad de una partícula es la siguiente:

\begin{equation}
\label{eq:pso_velocity}
v_i = c_0 \times v_i + c_1 \times R \times (B_i - P_i) + c_2 \times R \times (G - P_i)
\end{equation}

Donde $c_0, c_1, c_2$ son coeficientes de cuán importante es la velocidad $v$, la mejor posición personal $B_i$ y la mejor posición grupal $G$, respectivamente.
$R$ es un número aleatorio entre 0 y 1.

La ecuación para actualizar la posición de una partícula es:

\begin{equation}
\label{eq:pso_position}
P_i = P_i + v_i
\end{equation}

Donde $P_i$ es la posición actual, y $v_i$ es la velocidad recién calculada.

De manera empírica se ha comprobado que cuando la suma de $c_1 + c_2 = 4$ se alcanzan buenos resultados.
De igual manera, hay que limitar el movimiento máximo de una partícula o podría pasar por alto algunas soluciones importantes, por lo que existe un parámetro $v_{max}$.
Valores muy altos o muy bajos de $v_{max}$ pueden afectar la rapidez de convergencia o bien afectar la optimalidad de los resultados.
\end{document}