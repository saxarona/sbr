\documentclass[titlepage, letterpaper, fleqn]{article}
\usepackage[utf8]{inputenc}
\usepackage{fancyhdr}
\usepackage{amsmath}
\usepackage{extramarks}
\usepackage{enumitem}
\usepackage{multicol}
\usepackage{amssymb}
\usepackage{booktabs}
\usepackage{caption}
\usepackage{csquotes}
\usepackage{lmodern}
\usepackage[T1]{fontenc}

\topmargin=-0.45in
\evensidemargin=0in
\oddsidemargin=0in
\textwidth=6.5in
\textheight=9.0in
\headsep=0.25in


%
% You should change this things~
%

\newcommand{\mahteacher}{Dr. Santiago Enrique Conant Pablos}
\newcommand{\mahclass}{Sistemas Inteligentes}
\newcommand{\mahtitle}{Actividad 01}
\newcommand{\mahdate}{19 de agosto de 2016}
\newcommand{\spacepls}{\vspace{5mm}}
\newcommand{\qed}{\displaystyle{\, \, \blacksquare}}

\renewcommand{\familydefault}{\sfdefault} % sans-serif font?
\renewcommand{\baselinestretch}{1.3} % interlined
\bibliographystyle{abbrv} % Bibliography style?
\setlength\parindent{0pt} % Paragraph indentation
\setlength\parskip{1.5ex} % Space after/before paragraph

%
% Header markings
%

\pagestyle{fancy}
\lhead{01170065 - Xavier Sánchez}
\chead{}
\rhead{01299000 - Erim Sezer}
\lfoot{}
\rfoot{}


\renewcommand\headrulewidth{0.4pt}
\renewcommand\footrulewidth{0.4pt}

%
% Alias for the new definition section
%

\newcommand{\defy}[1]{
	\textbf{#1} % no space
}

\newcommand{\defys}[1]{
	\spacepls % with space
	\textbf{#1}
}

%
% My actual info
%

\title{
  \vspace{1in}
  \textbf{Tecnológico de Monterrey} \\
  \vspace{0.5in}
  \textmd{\mahclass} \\
  \large{\textit{\mahteacher}} \\
  \vspace{0.5in}
  \textsc{\mahtitle}
  \author{01170065 - MIT \\
  Xavier Sánchez \\
  \texttt{xavier.sanchezdz@gmail.com}
  \and
  01299000 - MSM \\
  Erim Sezer \\
  \texttt{erimsezer@gmail.com}
  }
  \date{\mahdate}
}

\begin{document}

\begin{titlepage}
\maketitle
\end{titlepage}

%
% Actual document starts here~
%

\section{Artificial Intelligence}
\subsection{Introduction}

Currently, humans are the most powerful and intelligent beings on the planet through a long history of evolution.
However, many scientists are asking themselves the same question: Can we build machines that are able to exceed human intelligence, one day?
Therefore, many institutions and scientists are showing a huge effort to answer this question.
Super-intelligence is basically a hypothetical agent with an intellect that is way smarter than the best human brain in various fields, such as scientific creativity, general wisdom and social skills.

\subsection{Does singularity really happen? Why?}
In general, technical singularity is the invention of artificial super-intelligence that will push technological advances rapidly, which will result in unseen and unpredictable changes for the human civilization.
More specifically, an intelligent agent that is able to upgrade itself, perform self-improvement and generate more intelligent agents is the cause of this intelligent explosion.

Consequently, super-intelligence through the intelligent explosion will easily and rapidly overstep human intelligence.
Nowadays, singularity is becoming more and more of an important topic. Since computers and technologies increases in power, humans are able to build machines that exceeds their own intelligence.
However, singularity has not happened yet.

Nevertheless, Ray Kurzweil, computer scientist and author, predicts that in 2045 machine intelligence will be more powerful than all human intelligence combined.
The theory of Ray Kurzweil isn't that unbelievable and so far away, since we are already in the area of cognitive hardware and brain-inspired architectures.

For example a study by the University of California San Diego and the University of Toronto found that a computer can detect false faces better than humans.
Examples such as Siri, Watson and Deep Blue are showing how close we are to singularity~\cite{Carter15,Utopia13}.

\subsection{If this happens, who is in the hands of whom? Why?}

Many futurists are predicting that singularity will shape human destiny once artificial intelligence surpasses human intelligence.
In fact, it will have a scary impact, but at the same time, exciting consequences.

Erich Horvitz, computer scientist, said that us humans won't lose control of certain kinds of intelligence.
He also mentioned that humans will be able to get incredible benefits from machine intelligence in many aspects of our life: from science to education, economics and daily life.

However, Professor Stephen Hawking showed his fear about the rapid advances of AI, since he believes that technology will become one day self-aware and consequently, the development of full artificial intelligence could bring the end of human race.

Finally, Elon Musk mentioned that we have to be careful while building artificial intelligence, and should do some regulations in order to later have control about the machines, and not vice versa.

As seen, it is very difficult to say who will control whom, however, the development of artificial intelligence brings many risks and challenges, which importance lies in setting clear regulations in order not to lose the control over the machines~\cite{Bostron98,Bostron03}.

\subsection{How to build an advanced intelligence that is benevolent and useful, and does not control or destroy us?}

The creation of a friendly AI needs significant design features and cognitive architecture in order to produce a benevolent AI.
The importance of creating a friendly AI lies in the distinction between AI and human psychology.

In general, benevolent AI means that AI systems won't harm humans and will perform their tasks for human well-being.
Friendly AI needs to be designed from scratch, so that designers could as well recognize that their own design might be flawed or wrong, and that could trigger machine learning and evolution over time in the wrong direction.
In addition, it is important that the goals we insert into machines (and their whole system motivation) are human friendly~\cite{Yudkowsky01}. 

\subsection{Can we encase (limit and control) the advancement of Artificial intelligence?}

As industry is concerned, AI is here but not evenly distributed.
Google, IBM and Facebook are all-in around AI topics.
It's because of these big-industries in AI, that it is almost impossible to stop the advancement of AI since the world is not a static environment.
Rapid advances in technologies and the rising amount of data will enhance the development of AI significantly and profoundly.

Nevertheless, Alexandru Tugui mentions that artificial intelligence must take into account the law of entropy,
that the entire foundation of artificial intelligence is based on informatics procedures;
and that the truth values (\textit{true} and \textit{false}) are major borders in artificial intelligence.
Artificial intelligence is based very much on symbolic logic, and has not succeeded in involving the so-called affective logic~\cite{Tugui04}.

\subsection{Should we delay or accelerate the development of these technologies? Why?}

As technologists, we think AI should be further studied and improved. The development of super-intelligence will improve the quality of human life in many ways.
As well, technology has created in the recent years many jobs and quality of life improvements.

Finally, trying to stop the creation of AI is impossible because software with intelligent capabilities is valuable to most people.
Creating intelligent software is a task someone will be always willing to pay for.

It is important to make it clear if super-intelligence will be feasible for humanity or won't be at all.
If super-intelligence will benefit humanity, then it will sooner or later be developed. However, if it is not, than AI will be a serious threat to the humanity~\cite{Urban15,Urban15B}.

\section{Concept Definitions}

The following concepts are relevant when talking about AI and Intelligent Systems. A short definition is given, as well as an example that further develops the concept.

\defy{Autonomous Agent}

An autonomous agent is an entity that makes its own choices, with no influence of a leader or a global plan.
Autonomous agents need a way to perceive the environment and to process its information in order to calculate an action~\cite{Shiffman}.

An example of an autonomous agent would be a human being. They have a way to perceive, process information and take decisions. In principle, all human beings are autonomous.

\defys{Strong AI vs Weak AI}

Weak AI is defined as the hypothesis that machines could act as if they thought.

A strong AI is, on the other hand, the hypothesis that machines actually think, not just simulate the thinking process~\cite{StrongAI}.

A weak AI example would be any piece of adaptive software, simulating learning.

A strong AI example can only be found in movies, like an android that is actually sentient, aware of its existence and its actions.

\defys{P Problem vs P Problem}

In computer science, problems are classified according to their complexity, depending on how much time is actually needed to solve them.

A problem is said to be a \textbf{P} problem (polynomial time) if it's solved in polynomial time. This means that there is at least one algorithm to solve it, such that the number of steps of the algorithm is bounded by a polynomial in \(n\), where \(n\) is the number of inputs for the problem~\cite{PProblem}.

A problem is said to be a \textbf{NP} problem if it is solvable in polynomial time by a non-deterministic (parallel) Turing machine~\cite{NPProblem}. If the solution for an \textbf{NP} problem is known, its correctness can be verified in \textbf{P} time.

Any arithmetical operation is considered a \textbf{P} problem. On the other hand, solving a jigsaw puzzle is considered as a \textbf{NP} problem.

\defys{Computational Intelligence}

Computational Intelligence is defined in~\cite{CI} as the study of the design of intelligent agents.

Poole et al. highlight the importance of building intelligent machines, and not simply machines that mimic human behavior.

Exercise 1 was a brief example of computational intelligence.

\pagebreak

\defys{Markov Decision Process}

A Markov Decision Process (MDP) is a sequential decision problem for a stochastic environment that is fully observable, that has a Markovian transition model (in which reaching a state only depends on the previous state and nothing else) and also has additive rewards~\cite{MDP}.

An MDP consists formally of a set of states \(S\) with an initial state \(s_0\), a set of actions \(A(s)\) for each state, a transition model \(P(s' \mid s,a)\) and a reward function \(R(s)\).

Many planning activities can be represented using MDPs, like planning on how many crops to plant and harvest based on the weather and soil states.

\defys{Cognitive Science}

According to the Stanford Encyclopedia of Philosophy in~\cite{CogSci}, Cognitive Science is an interdisciplinary study of the mind and intelligence that embraces philosophy, psychology, artificial intelligence, neuroscience, linguistics and anthropology.

It started in the mid-1950s, and consolidated in the mid 1970s with the beginning of the Cognitive Science Society and its own journal.

\defys{Heuristic}

Literally, heuristic is an adjective that describes anything that serves as an aid to learning, discovery or problem-solving by experimental methods.

Heuristic, however, can also be used as a noun, which refers to a heuristic procedure or technique~\cite{heur}.

The ``greedy'' search is an heuristic that chooses the higher value of all available to find a solution, and it's a good example of a heuristic.

\defys{Technological singularity}

Vernor Vinge, mathematics professor and fiction author, defined technological singularity as the point in history where an ultra-intelligent machine could design even better machines, where the intelligence of man would be left far behind.

The curve of technological progress will be of near-infinite growth, instead of showing an S-shaped curve as most technologies have shown so far~\cite{TechSing}.

Matrix, the movie, is a fine example of what could happen if an ultra-intelligent machine is designed.

\defys{Big Data}

International Business Machines (IBM) defines Big Data~\cite{Big} as the following:

\begin{displayquote}
  Data that is being generated by everything around us at all times.
  Every digital process and social media exchange produces it.
  Systems, sensors and mobile devices transmit it.
  Big data is arriving from multiple sources at an alarming velocity, volume and variety.
  To extract meaningful value from big data, optimal processing power, analytics capabilities and skills are needed.
\end{displayquote}

Because the amount of data has grown immensely since the Internet era, it has received the ``Big'' adjective.

Click-streams in videos, websites and ads are analyzed with big data analysis tools to detect patterns and improve target reach.

\defys{Ontology}

When designing an intelligent agent, the knowledge engineering process consists of certain steps.
One of these steps is to translate the important domain-level concepts into logic-level names.
This vocabulary is known as the ontology of the domain:
a particular theory of the nature of existence, what exists in the world designed~\cite{Onto}.

The PEAS description of an agent can be seen as the ontology of its world.

\section{Intelligent Agents formulation}
\label{sec:agent}

The intelligent agent proposed in this section is an administrative helper which aids the user to plan, manage and complete research tasks efficiently by letting them focus on things that do matter.
This administrative AI will be called \textbf{ALiCE}, which stands for Administrative Liaison Controlled Environment.
With ALiCE, Petty tasks like handling meetings and setting deadlines, file management, document formatting or data analytics on the ongoing project will need less attentions from the user.
Via behavior analysis, ALiCE will learn which things matter and which don't, as well as when is the best moment to remind the user of important events or upcoming reports.

\textbf{Performance} will be measured with both explicit feedback (via satisfaction surveys) and by analyzing job completion (progress per hour, time invested on tasks, idle time).

The \textbf{environment} will consist of all the jobs, tasks, events and goals the user create, as well as all the files in the working directory.

All ALiCE's functions and features will be its \textbf{actuators}. ALiCE will be modifying the environment by displaying notifications on the screen, writing files and sending emails through the network.

ALiCE will receive information about the environment via keystrokes, click and touch gestures, and some voice commands. This means that the keyboard, mouse (or touchscreen if in any mobile device) and microphone will be ALiCE's \textbf{sensors}.

Due its nature as a software package, ALiCE will be aware of all the environment described (files, emails, deadlines, etc).
All actions will modify this \textbf{fully-observable} environment, but not files outside ALiCE's reach/permissions nor tasks that were not described by the user, as these elements are not considered as part of the environment and ALiCE is not aware of them.

The environment will be considered as \textbf{strategic}, for all environment states will be determined by both the user's and ALiCE's actions as there are no stochastic or random events in it.
This is why the environment is also considered to be \textbf{dynamic}, as it is modified even without ALiCE's actions.

As a \textbf{learning agent}, ALiCE training and behavior analysis will consider all actions of the user.
This means that the environment will be \textbf{sequential}. A snapshot of the environment is meaningless by its own for ALiCE's recommendation system.

As most software bots, ALiCE works in a \textbf{discrete} environment, with limited actions and input commands. However, additional commands but not perceptions could also be created via user input.

Finally, the environment considers ALiCE as the only agent, so it's a \textbf{single-agent} environment.
This doesn't mean integration with other softbots is implausible in the future.

A \textbf{learning-agent architecture} is needed for ALiCE's core functions.
Performance standards will be updated using its critic element, by analyzing times, completion rates and via direct feedback (surveys).
This feedback will allow the learning element to generate changes on ALiCE's behavior, and thus influencing its performance.

In order to label an action as valid or as a good suggestion, ALiCE's problem generator element need to simulate situations in which it can expand its knowledge about the environment. The training sessions should include some of these situations in order for ALiCE to learn the user's preferences.

\bibliography{biblio}

\end{document}