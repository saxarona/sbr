\documentclass[titlepage, letterpaper, fleqn]{article}
\usepackage[utf8]{inputenc}
\usepackage{fancyhdr}
\usepackage{amsmath}
\usepackage{extramarks}
\usepackage{enumitem}
\usepackage{multicol}
\usepackage{amssymb}
\usepackage{booktabs}
\usepackage{caption}
\usepackage{hyperref}
\usepackage{csquotes}

\topmargin=-0.45in
\evensidemargin=0in
\oddsidemargin=0in
\textwidth=6.5in
\textheight=9.0in
\headsep=0.25in


%
% You should change this things~
%

\newcommand{\mahteacher}{Dr. Santiago Enrique Conant Pablos}
\newcommand{\mahclass}{Sistemas Inteligentes}
\newcommand{\mahtitle}{Actividad 01}
\newcommand{\mahdate}{19 de agosto de 2016}
\newcommand{\spacepls}{\vspace{5mm}}
\newcommand{\qed}{\displaystyle{\, \, \blacksquare}}

\renewcommand{\familydefault}{\sfdefault} % sans-serif font?
\bibliographystyle{abbrv} % Bibliography style?
\setlength\parindent{0pt} % Paragraph indentation
\setlength\parskip{1.5ex} % Space after/before paragraph
\renewcommand{\baselinestretch}{1.3}

%
% Header markings
%

\pagestyle{fancy}
\lhead{01170065 - Xavier Sánchez}
\chead{}
\rhead{01299000 - Erim Sezer}
\lfoot{}
\rfoot{}


\renewcommand\headrulewidth{0.4pt}
\renewcommand\footrulewidth{0.4pt}

%
% Alias for the new definition section
%

\newcommand{\defy}[1]{
	\textbf{#1} % no space
}

\newcommand{\defys}[1]{
	\spacepls % with space
	\textbf{#1}
}

%
% My actual info
%

\title{
    \vspace{1in}
    \textbf{Tecnológico de Monterrey} \\
    \vspace{0.5in}
    \textmd{\mahclass} \\
    \large{\textit{\mahteacher}} \\
    \vspace{0.5in}
    \textsc{\mahtitle}
    \author{01170065  - MIT \\
    Xavier Sánchez \\
    \texttt{xavier.sanchezdz@gmail.com}
    \and
    01299000 - MSM \\
    Erim Sezer \\
    \texttt{erimsezer@gmail.com}
    }
    \date{\mahdate}
}

\begin{document}

\begin{titlepage}
\maketitle
\end{titlepage}

%
% Actual document starts here~
%

\section{Definitions}
\label{sec:defs}

\defy{Autonomous Agent}

An autonomous agent is an entity that makes its own choices, with no influence of a leader or a global plan.
Autonomous agents need a way to perceive the environment and to process its information in order to calculate an action~\cite{Shiffman}.

An example of an autonomous agent would be a human being. They have a way to perceive, process information and take decisions. In principle, all human beings are autonomous.

\defys{Strong AI vs Weak AI}

Weak AI is defined as the hypothesis that machines could act as if they thought.

A strong AI is, on the other hand, the hypothesis that machines actually think, not just simulate the thinking process~\cite{StrongAI}.

A weak AI example would be any piece of adaptive software, simulating learning.

A strong AI example can only be found in movies, like an android that is actually sentient, aware of its existence and its actions.

\defys{P Problem vs P Problem}

In computer science, problems are classified according to their complexity, depending on how much time is actually needed to solve them.

A problem is said to be a \textbf{P} problem (polynomial time) if it's solved in polynomial time. This means that there is at least one algorithm to solve it, such that the number of steps of the algorithm is bounded by a polynomial in \(n\), where \(n\) is the number of inputs for the problem~\cite{PProblem}.

A problem is said to be a \textbf{NP} problem if it is solvable in polynomial time by a non-deterministic (parallel) Turing machine~\cite{NPProblem}. If the solution for an \textbf{NP} problem is known, its correctness can be verified in \textbf{P} time.

Any arithmetical operation is considered a \textbf{P} problem. On the other hand, solving a jigsaw puzzle is considered as a \textbf{NP} problem.

\defys{Computational Intelligence}

Computational Intelligence is defined in~\cite{CI} as the study of the design of intelligent agents.

Poole et al. highlight the importance of building intelligent machines, and not simply machines that mimic human behavior.

Exercise 1 was a brief example of computational intelligence.

\pagebreak

\defys{Markov Decision Process}

A Markov Decision Process (MDP) is a sequential decision problem for a stochastic environment that is fully observable, that has a Markovian transition model (in which reaching a state only depends on the previous state and nothing else) and also has additive rewards~\cite{MDP}.

An MDP consists formally of a set of states \(S\) with an initial state \(s_0\), a set of actions \(A(s)\)for each state, a transition model \(P(s' \mid s,a)\) and a reward function \(R(s)\).

Many planning activities can be represented using MDPs, like planning on how many crops to plant and harvest based on the weather and soil states.

\defys{Cognitive Science}

According to the Stanford Encyclopedia of Philosophy in~\cite{CogSci}, Cognitive Science is an interdisciplinary study of the mind and intelligence that embraces philosophy, psychology, artificial intelligence, neuroscience, linguistics and anthropology.

It started in the mid-1950s, and consolidated in the mid 1970s with the beginning of the Cognitive Science Society and its own journal.

\defys{Heuristic}

Literally, heuristic is an adjective that describes anything that serves as an aid to learning, discovery or problem-solving by experimental methods.

Heuristic, however, can also be used as a noun, which refers to a heuristic procedure or technique~\cite{heur}.

The ``greedy'' search is an heuristic that chooses the higher value of all available to find a solution, and it's a good example of a heuristic.

\defys{Technological singularity}

Vernor Vinge, mathematics professor and fiction author, defined technological singularity as the point in history where an ultra-intelligent machine could design even better machines, where the intelligence of man would be left far behind.

The curve of technological progress will be of near-infinite growth, instead of showing an S-shaped curve as most technologies have shown so far~\cite{TechSing}.

Matrix, the movie, is a fine example of what could happen if an ultra-intelligent machine is designed.

\defys{Big Data}

International Business Machines (IBM) defines Big Data~\cite{Big} as the following:

\begin{displayquote}
    Data that is being generated by everything around us at all times.
    Every digital process and social media exchange produces it.
    Systems, sensors and mobile devices transmit it.
    Big data is arriving from multiple sources at an alarming velocity, volume and variety.
    To extract meaningful value from big data, optimal processing power, analytics capabilities and skills are needed.
\end{displayquote}

Because the amount of data has grown immensely since the Internet era, it has received the ``Big'' adjective.

Click-streams in videos, websites and ads are analyzed with big data analysis tools to detect patterns and improve target reach.

\defys{Ontology}

When designing an intelligent agent, the knowledge engineering process consists of certain steps.
One of these steps is to translate the important domain-level concepts into logic-level names.
This vocabulary is known as the ontology of the domain:
a particular theory of the nature of existence, what exists in the world designed~\cite{Onto}.

The PEAS description of an agent can be seen as the ontology of its world.

\bibliography{biblio}

\end{document}